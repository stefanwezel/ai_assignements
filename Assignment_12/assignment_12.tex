% ----------------------- TODO ---------------------------
%Template
% !TeX spellcheck = de 
\documentclass[a4paper]{scrartcl}
\usepackage[utf8]{inputenc}
%\usepackage[ngerman]{babel}
\usepackage{geometry,forloop,fancyhdr,fancybox,lastpage}
\usepackage{listings}
\lstset{frame=tb,
	language=Java,
	aboveskip=3mm,
	belowskip=3mm,
	showstringspaces=false,
	columns=flexible,
	basicstyle={\small\ttfamily},
	numbers=left,
	numberstyle=\tiny\color{gray},
	keywordstyle=\color{blue},
	commentstyle=\color{dkgreen},
	stringstyle=\color{mauve},
	breaklines=true,
	breakatwhitespace=true,
	tabsize=3
}
\geometry{a4paper,left=3cm, right=3cm, top=3cm, bottom=3cm}
% Diese Daten müssen pro Blatt angepasst werden:
\newcommand{\NUMBER}{6}
\newcommand{\EXERCISES}{3}
% Diese Daten müssen einmalig pro Vorlesung angepasst werden:
\newcommand{\COURSE}{Algorithmen}
\newcommand{\TUTOR}{TBD}
\newcommand{\STUDENTA}{}
\newcommand{\STUDENTB}{Lukas Guenthner}
\newcommand{\STUDENTC}{Gwent Krause}
\newcommand{\DEADLINE}{07.06.2018}




%Math
\usepackage{amsmath,amssymb,tabularx}

%Figures
\usepackage{graphicx,tikz,color,float}
\graphicspath{ {home/stefan/picures/} }
\usetikzlibrary{shapes,trees}

%Algorithms
\usepackage[ruled,linesnumbered]{algorithm2e}

%Kopf- und Fußzeile
\pagestyle {fancy}
\fancyhead[L]{Tutor: \TUTOR}
\fancyhead[C]{\COURSE}
\fancyhead[R]{\today}

\fancyfoot[L]{}
\fancyfoot[C]{}
\fancyfoot[R]{Seite \thepage}

%Formatierung der Überschrift, hier nichts ändern
\def\header#1#2{
	\begin{center}
		{\Large\bf Übungsblatt #1}\\
		{(Abgabetermin #2)}
	\end{center}
}

%Definition der Punktetabelle, hier nichts ändern
\newcounter{punktelistectr}
\newcounter{punkte}
\newcommand{\punkteliste}[2]{%
	\setcounter{punkte}{#2}%
	\addtocounter{punkte}{-#1}%
	\stepcounter{punkte}%<-- also punkte = m-n+1 = Anzahl Spalten[1]
	\begin{center}%
		\begin{tabularx}{\linewidth}[]{@{}*{\thepunkte}{>{\centering\arraybackslash} X|}@{}>{\centering\arraybackslash}X}
			\forloop{punktelistectr}{#1}{\value{punktelistectr} < #2 } %
			{%
				\thepunktelistectr &
			}
			#2 &  $\Sigma$ \\
			\hline
			\forloop{punktelistectr}{#1}{\value{punktelistectr} < #2 } %
			{%
				&	
			} &\\
			\forloop{punktelistectr}{#1}{\value{punktelistectr} < #2 } %
			{%
				&
			} &\\
		\end{tabularx}
	\end{center}
}

\begin{document}
	
	\begin{tabularx}{\linewidth}{m{0.2 \linewidth}X}
		\begin{minipage}{\linewidth}
		%	\STUDENTA
			\STUDENTB
			\STUDENTC
		\end{minipage} & \begin{minipage}{\linewidth}
			\punkteliste{1}{\EXERCISES}
		\end{minipage}\\
	\end{tabularx}
	
	%\header{Nr. \NUMBER}{\DEADLINE}
	
	% ----------------------- TODO ---------------------------
	% Hier werden die Aufgaben/Lösungen eingetragen:
	
\section*{Question 1:}

\section*{Question 3:}
\begin{tabular}{|c|c|c|c|c|c|}
	\hline 
	S & E & A & $S \Rightarrow (\neg A \Rightarrow \neg E)$ & $S \Leftrightarrow (A \wedge E)$ & $S  \Rightarrow(E\Leftrightarrow A)$ \\ 
	\hline 
	0 & 0 & 0 & 1 & 1 & 1 \\ 
	\hline 
	1 & 0 & 0 & 1 & 0 & 1 \\ 
	\hline 
	0 & 1 & 0 & 1 & 0 & 1 \\ 
	\hline 
	1 & 1 & 0 & 0 & 1 & 0 \\ 
	\hline 
	0 & 0 & 1 & 1 & 0 & 1 \\ 
	\hline 
	1 & 0 & 1 & 1 & 1 & 0 \\ 
	\hline 
	0 & 1 & 1 & 1 & 0 & 1 \\ 
	\hline 
	1 & 1 & 1 & 1 & 1 & 1 \\ 
	\hline 
\end{tabular} 
\end{document}
%%% Local Variables:
%%% mode: latex
%%% TeX-master: t
%%% End:
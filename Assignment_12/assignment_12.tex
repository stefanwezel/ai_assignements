% ----------------------- TODO ---------------------------
%Template
% !TeX spellcheck = de 
\documentclass[a4paper]{scrartcl}
\usepackage[utf8]{inputenc}
%\usepackage[ngerman]{babel}
\usepackage{geometry,forloop,fancyhdr,fancybox,lastpage}
\usepackage{listings}
\lstset{frame=tb,
	language=Java,
	aboveskip=3mm,
	belowskip=3mm,
	showstringspaces=false,
	columns=flexible,
	basicstyle={\small\ttfamily},
	numbers=left,
	numberstyle=\tiny\color{gray},
	keywordstyle=\color{blue},
	commentstyle=\color{dkgreen},
	stringstyle=\color{mauve},
	breaklines=true,
	breakatwhitespace=true,
	tabsize=3
}
\geometry{a4paper,left=3cm, right=3cm, top=3cm, bottom=3cm}
% Diese Daten müssen pro Blatt angepasst werden:
\newcommand{\NUMBER}{6}
\newcommand{\EXERCISES}{4}
% Diese Daten müssen einmalig pro Vorlesung angepasst werden:
\newcommand{\COURSE}{Artificial Intelligwents}
\newcommand{\TUTOR}{TBD}
\newcommand{\STUDENTA}{}
\newcommand{\STUDENTB}{Stefan Wezel}
\newcommand{\STUDENTC}{Gwent Krause}
\newcommand{\DEADLINE}{07.06.2018}




%Math
\usepackage{amsmath,amssymb,tabularx}

%Figures
\usepackage{graphicx,tikz,color,float}
\graphicspath{ {home/stefan/picures/} }
\usetikzlibrary{shapes,trees}

%Algorithms
\usepackage[ruled,linesnumbered]{algorithm2e}

%Kopf- und Fußzeile
\pagestyle {fancy}
\fancyhead[L]{Tutor: \TUTOR}
\fancyhead[C]{\COURSE}
\fancyhead[R]{\today}

\fancyfoot[L]{}
\fancyfoot[C]{}
\fancyfoot[R]{Seite \thepage}

%Formatierung der Überschrift, hier nichts ändern
\def\header#1#2{
	\begin{center}
		{\Large\bf Übungsblatt #1}\\
		{(Abgabetermin #2)}
	\end{center}
}

%Definition der Punktetabelle, hier nichts ändern
\newcounter{punktelistectr}
\newcounter{punkte}
\newcommand{\punkteliste}[2]{%
	\setcounter{punkte}{#2}%
	\addtocounter{punkte}{-#1}%
	\stepcounter{punkte}%<-- also punkte = m-n+1 = Anzahl Spalten[1]
	\begin{center}%
		\begin{tabularx}{\linewidth}[]{@{}*{\thepunkte}{>{\centering\arraybackslash} X|}@{}>{\centering\arraybackslash}X}
			\forloop{punktelistectr}{#1}{\value{punktelistectr} < #2 } %
			{%
				\thepunktelistectr &
			}
			#2 &  $\Sigma$ \\
			\hline
			\forloop{punktelistectr}{#1}{\value{punktelistectr} < #2 } %
			{%
				&	
			} &\\
			\forloop{punktelistectr}{#1}{\value{punktelistectr} < #2 } %
			{%
				&
			} &\\
		\end{tabularx}
	\end{center}
}

\begin{document}
	
	\begin{tabularx}{\linewidth}{m{0.2 \linewidth}X}
		\begin{minipage}{\linewidth}
		%	\STUDENTA
			\STUDENTB\\
			\STUDENTC\\
		\end{minipage} & \begin{minipage}{\linewidth}
			\punkteliste{1}{\EXERCISES}
		\end{minipage}\\
	\end{tabularx}
	
	%\header{Nr. \NUMBER}{\DEADLINE}
	
	% ----------------------- TODO ---------------------------
	% Hier werden die Aufgaben/Lösungen eingetragen:
	
\section*{Question 1:}
\includegraphics*[scale = 0.4]{AIQ1.png}
\section*{Question 3:}
\subsection*{a)}
\begin{tabular}{|c|c|c|c|c|c|}
	\hline 
	S & E & A & $S \Rightarrow (\neg A \Rightarrow \neg E)$ & $S \Leftrightarrow (A \lor E)$ & $S  \Rightarrow(E\Leftrightarrow A)$ \\ 
	\hline 
	0 & 0 & 0 & 1 & 1 & 1 \\ 
	\hline 
	1 & 0 & 0 & 1 & 0 & 1 \\ 
	\hline 
	0 & 1 & 0 & 1 & 0 & 1 \\ 
	\hline 
	1 & 1 & 0 & 0 & 1 & 0 \\ 
	\hline 
	0 & 0 & 1 & 1 & 0 & 1 \\ 
	\hline 
	1 & 0 & 1 & 1 & 1 & 0 \\ 
	\hline 
	0 & 1 & 1 & 1 & 0 & 1 \\ 
	\hline 
	1 & 1 & 1 & 1 & 1 & 1 \\ 
	\hline 
\end{tabular}\\
\subsection*{i)}
Stimmt nicht mit der Aussage des Professors ueberein. Nach dieser Aussage kann ein Nicht-Student eine finale Pruefung vorstellen, wenn er mindestens 5 Abgaben bearbeitet hat.\\
\subsection*{ii)}
Stimmt nicht mit der Aussage des Proffesors ueberein. Es koennte ein Student seine finale Pruefung vorstellen ohne die noetigen 5 Abgaben bearbeitet zu haben.\\
\subsection*{iii)}
Stimmt nicht mit der Aussage des Professors ueberein. Ein Student muss wenn er die Abgaben erledigt hat auch an der finalen Pruefung teilnehmen und hat nicht nur die Erlaubnis dazu.\\ 

\subsection*{b)}
\subsubsection*{i)}
CNF:\\
$A \lor \neg E \lor \neg S$\\
Hornklausel, da nur A nicht negiert ist.\\

\subsubsection*{ii)}
CNF:\\
$(A \lor E \lor \neg S) \land (A \lor \neg E \lor S) \land (\neg A \lor E \lor S) \land (\neg A \lor \neg E \lor S)$\\
Nur $(\neg A \lor \neg E \lor S)$ Hornklausel, bei den anderen Klauseln sind mehr als eine Variable nicht negiert.\\
\subsubsection*{iii)}
CNF:\\
$(A \lor \neg E \lor \neg S) \land (\neg A \lor \neg E \lor S)$\\
Hornklauseln, da bei beiden Klauseln nicht mehr als eine Variable nicht negiert ist.\\




\section*{Question 4}
\subsection*{(a)}
$\exists o\; Occupation(Matt, o) \land \neg Occupation(Matt, student)$


\subsection*{(b)}
$\forall p_1 : Patient(Matt, p_1)$

\subsection*{(c)}
$\exists p_1 \forall p_2 Patient(p_1, p_2) \land Occupation(p_1, Researcher)$


\subsection*{(d)}
$\exists Boss(Jane, p_2) \land \neg Occupation(p_2, doctor)$


\subsection*{(e)}



\subsection*{(f)}
$Occupation(Jane, physicist) \lor Occupation(Jane, Doctor)$





\end{document}
%%% Local Variables:
%%% mode: latex
%%% TeX-master: t
%%% End: